\documentclass[12pt]{tdtp}
\usepackage{hyperref}
\usepackage{tabularx,colortbl}
\usepackage{multirow}
\usepackage{listings}
\lstset{
	language=VHDL,
basicstyle=\tiny\ttfamily}
\definecolor{light-gray}{gray}{0.96}
\definecolor{pageheading-gray}{gray}{0.2}
\definecolor{dark-gray}{gray}{0.45}
\definecolor{dark-green}{rgb}{0.245,0.121,0.0}

\newcommand{\auteur}{Cedric Lemaitre}
\newcommand{\couriel}{c.lemaitre58@gmail.com}
\newcommand{\promo}{BScv}
\newcommand{\annee}{2017-2018}
\newcommand{\matiere}{Computer science}

\newcommand{\tdtp}{Lab \#1}
\renewcommand{\sujet}{Function, array, matrix}


\begin{document}
\titre
\textit{NB : use good practice for naming and write code \footnote{\url{https://google.github.io/styleguide/cppguide.html}}!!!}

%%%%%%%%%%%
\Exo


Write in c++, codes which correpond to the following problem :

\begin{enumerate}
	\item Permutate 2 variables which have been input by keyboard
	\item Test if 2 numbers inupt from keyboard have the same sign
	\item Test if 2 numbers inupt from keyboard are even or odd
	\item Print in the terminal 1 times the word "First"
	\item Print in the terminal 3 times the word "First" or 2 times the word "second" depending of the user choice
\end{enumerate}

%%%%%%%%%%%%
\Exo


Display in the terminal the following function :

\begin{eqnarray}
	f(x) = \sin(x) + \ln(x)
\end{eqnarray}

where $x \in [1:10]$

%%%%%%%%%%%
\Exo 

Write 2 versions of a function which to swap 2 integers. The first one have to use \textit{int *} and the second will use \textit{int \&}. 
%%%%%%%%%%
\Exo


Create functions which allows to make the following sort :

\begin{enumerate}
	\item Buble sort : \url{https://en.wikipedia.org/wiki/Bubble_sort}
	\item Quick sort : \url{https://en.wikipedia.org/wiki/Quicksort}	
\end{enumerate}

For each one evaluate the running time and the display it.


test{NB : Use array allocate dynamicly}

%%%%%%%%%%
\Exo

In this part we will manipulate Array and Matrix :
\begin{enumerate}
	\item Transpose a matrix
	\item Transform a matrix in a mono dimensional array.
	\item Compute, store and display the N first line of Pascal Triangle	
\end{enumerate}

%%%%%%%%%

\end{document}
